%        File: MY.tex
%     Created: Don Feb 10 01:00  2011 C
% Last Change: Don Feb 10 01:00  2011 C
%
\documentclass[a4paper]{article}

\usepackage{ngerman}
\usepackage[utf8]{inputenc}
\usepackage[]{graphicx}
\usepackage[]{amsmath}

\begin{document}
\section{Limitierende Reaktanten - Stöchimetrie}
\subsection{}
\begin{align}
  Fe_2O_3 + 3 CO &\rightarrow 2Fe + 3CO_2\\
  Fe_2^{+3}O_3^{-2} + 3 C^{+2}O^{-2} &\rightarrow 2Fe + 3C^{+4}O_2^{-2}\\
  M_{Fe_2O_3}=2*56+3*16=160\\
  M_{Fe}=56\\
  \frac{160}{2*56}=\frac{150}{x} \Rightarrow x=\frac{150 * 2 * 56}{160}=105g\\
  y=\frac{87.9}{105}=84\%
\end{align}

\subsection{}
\begin{align}
  Na_2SiO_3 + 8 HF \rightarrow H_2SiF_6 + 2 NaF +3 H_2O\\
  n_{HF}=0.3*8=2.4 Mol\\
  M_{HF}=1+19=20 \rightarrow n_{HF}=0.5 Mol\\
  M{Na_2SiO_3}=2*23+28+3*16=122 \rightarrow n_{Na_2SiO_3}=0.7 Mol\\
  \text{HF wirkt begrenzend}\\
  M_{NaF}=23+19=42\\
  \frac{2*42}{8*20}=\frac{x}{10} \Rightarrow x=5.25g
\end{align}

\subsection{}
\begin{align}
  2 H_2^{+1}C_2^{-1} + 5 0_2 \rightarrow 4 C^{+4}O_2^{-2} + 2 H_2^{+1}O^{-2}\\
  M_{H_2C_2}=2+2*12=26 \rightarrow n_{H_2C_2}=10/26=0.385 Mol\\
  M_{O_2}=32 \rightarrow n_{O_2}=10/32=0.3125 Mol\\
  M_{CO_2}=12+32=44\\
  M{H_2O}=18\\
  \text{O2 is wirkt begrenzend}\\
  n_{O_2}=0\\
  n_{H_2P_2}=10-(\frac{2}{5}0.3125)*26=6.76g\\
  n_{CO_2}=\frac{4}{5}*0.3125*44=11g\\
  n_{H_2O}=\frac{2}{5}*0.3125*18=2.25g
\end{align}

\subsection{}
\begin{align}
  K_3PO_4 \dots \text{Kaliumnposphat}\\
  AgNO_3 \dots \text{Silbernitrat}\\
  K_3^{+1}P^{+5}O_4^{-2} + 3 Ag^{+1}N^{+5}O_3^{-2} \rightarrow Ag_3^{+1}PO_4^{-3} + 3 K^{+1}NO_3^{-1}\\
  M_{K_3PO_4}=3*39+31+4*16=212 \rightarrow n_{K_3PO_4}=\frac{70.5*10^-3}{212}=0.33 10^{-3}Mol\\
  M_{AgNO_3}=108+14+3*16=170 \rightarrow n_{AgNO_3}=0.4*15*10^{-3}=6*10^{-3}Mol\\
  \text{K3PO4 ist begrenzend}\\
  m_{Ag_3PO_4}=6*10^{-3}*(108+31+4*16)=1.218g\\
  L = [A]^a*[B]^b = [Ag]^3+[PO_4] 
\end{align}

\subsection{}
\begin{align}
  Ca^{+2}(OH)_2 + 2 H^{+1}Br^{-1} \rightarrow CaBr_2 + 2 H_2O\\
  n_{HBr}=5*10^{-2}*48.8*10^{-3}= 0.0025Mol\\  
  n_{Ca(OH)_2}=n_{HBr}*0,5 = 0,00125 Mol\\
  [Ca(OH)_2]=\frac{0,00125}{0,1}=0,0123 Mol/L\\
  \Rightarrow 0,0123*(17*2+40)=0,91 g/L
\end{align}

\section{Thermochemie}
\subsection{}
\begin{align}
  E_{4g}=7,794*(39.5-25)= 113 kJ\\
  M_{CH_6N_2}=12+6+28=46g/Mol\\
  E_{1mol}=E_{4g}*\frac{46}{4}=1300 kJ
\end{align}

\subsection{}
\begin{align}
  2 C_6H_6 + 15 O_2 \rightarrow 12 CO_2 + 6 H_2O\\
  \Delta H = -12*393.5 - 6*285,5 - 2*49 = -6533 kJ\\
  \Rightarrow E = \frac{\Delta H}{2*6*13}=-40,3 kJ/1g
\end{align}

\subsection{}
\begin{align}
  CH_4 + 2 O_2 \rightarrow CO_2 + 2 H_2O\\
  \Delta H = -393,5 -2*285,8+74,8 = -890,3 kJ/Mol\\
  E = \frac{19}{16}*-890,3 = -1057,2kJ
\end{align}

\subsection{}
\begin{align}
  C + 0_2 \rightarrow CO_2\\
  \Delta H = -393,5kJ\\
  E = \frac{13000}{12}*-393,5=-426291,7kJ
\end{align}

\subsection{}
\begin{align}
  C_2H_5OH + 3 O_2 -> 2 CO_2 + 3 H_2O\\
  H^0_f(C_2H_5OH)=-(-1367+392,5*2+3*285,8)=-275,4kJ
\end{align}

\section{Periodische Eigenschaften - Chemische Bindung - Bindungstheorie}
\subsection{}
Je größer der Radius des Atoms desto kleiner ist die Ionisierungsenergie
Je mehr Kernladungen das Atom hat desto größer ist die Ionisierungsenergie
Hochste : He
Niedrigste : Fr

\subsection{}
\[[Ar]3d^9 +2e^- = [Ar]3d^{10}4s^1 = Cu\]
\[[Xe]4f^{14}5d^{10}6s^{2} + e^- = [Xe]4f^{14}5d^{10}6s^{2}6p^1 = Tl\]
beide keine Freien Elektronen 

\subsection{}
  Da Kalzium in eriner höheren Periode als Magnnesium ist, ist es weniger reaktiv.
  Da Kalium nur 1 Valenzelektron hat ist es reaktiver als Kalzium.

\subsection{}
  O<Br<K<Mg

  Nichtmetall(Sauerstoffgruppe)<Nichtmetall(Halogen)<Metall(Alkalimetall)<Metall(Erdalkali)

\subsection{}
\begin{align}
  M_C=12 \\
  M_O=16\\
  M_{Cl}=35,5 \\
  m_C=98,9*0,1214=12g\\
  m_O=98,9*0,1617=16g\\
  m_{Cl}=98,9*0,7169=71g\\
  n_C=1\\
  n_O=1\\
  n_{Cl}=2\\
  \Rightarrow COCl_2\\
  (Cl-)^2C=O
\end{align}

\subsection{}
\begin{align}
  \text{Annahme 100g}\\
  n_H=frac{0,1372*100}{1}=13,72Mol\\
  n_C=\frac{0,6813*100}{12}=5,6775Mol\\
  n_O=\frac{0,1815*100}{16}=1,134375Mol\\
  \text{Division durch die kleinste Molzahl}
  \Rightarrow C_5H_{12}O\\
\end{align}

\subsection{}
  Ethylen($C_2H_4$) hat 4 $\sigma$ und eine $\pi$ Bindung

\subsection{}
  $O_2$ Molekülschema befindet sich in den Folien

\subsection{}
\begin{align}
  n_H=\frac{50*0,021}{1}=1\\
  n_N=\frac{50*0,298}{14}=1\\
  n_O=\frac{50*0,681}{16}=2\\
  \rightarrow NHO_2(O=N-O-H)\\
  sp^2-Hybrit:2*\sigma \quad 1*\pi
\end{align}

\section{Gase \& Flüssigkeiten \& Lösungen}
\subsection{}
\begin{align}
  n_{0_2}=\frac{6}{32}=0,1875Mol\\
  n_{CH_4}=\frac{9}{16}=0,5625Mol\\
  p_{O_2}=\frac{0,1875*0,0821*273}{0,1}=42 atm\\
  p_{CH_4}=\frac{0,5625*0,0821*273}{0,1}=126 atm\\
  p_{ges}=168 atm
\end{align}

\subsection{}
\begin{align}
  n_{N_2}=\frac{p*V}{R*T}=\frac{745*0,511}{62,36*(273+26)}=0,02 Mol\\
  m_{NH_4NO_2}=(2*14+4+32)*0,02=1,28g
\end{align}

\subsection{}
\begin{align}
  n_C=\frac{0,857*1.56}{12}=0,11141 Mol\\
  n_H=\frac{0,143*1,56}{1}=0,22308 Mol\\
  n = \frac{0,984*1}{0,0821*(273+50)}=0,0371 Mol\\
  \frac{1,56}{0,0371}=42g/Mol\\
  42=(12+2)x \Rightarrow C_3H_{6}
  \text{ideales Gas - Punkte - kein Volumen - Argon - Edelgas}
\end{align}

\subsection{}
\begin{align}
  MgCO_3 + CaO + HCl \rightarrow CO_2 + MgO + CaCl + H_2O  
\end{align}

\subsection{}
Phasendiagramm ;)

\subsection{}
a. Dipolkräfte,Dispersionskräfte
b. Van da Waals Kräfte
c. Elektromagnetische Anziehung (Ionen)
d. Elektromagnetische Anziehung (Gas - Rumof)

\subsection{}
amorph \dots ungeordnet, Glas
kristallin \dots geordnet, dichter, spröder, Quarz

\subsection{}
\begin{align}
  m_{C_3H_5(OH)_3}=1.26*50=63g\\
  n_{C_3H_5(OH)_3}=\frac{63}{3*12+8+3*16}=0,674Mol
  n_{H_2O}=500/18=27,8
  p=23,8*\frac{27,8}{27,8+0,674}= 23,24 torr
\end{align}

\subsection{}
\begin{align}
  p=p_{H_2O}*X(H_2O)\\
  \frac{x}{x+y}=X(H_2O)\\
  \frac{x}{X(H_2O)}=y+x \quad y=\frac{x}{X(H_2O)}-x\\
  X(ethylen)=1-X(H_2O)=1-\frac{p}{p_{H_2O}}=0,29
\end{align}

\subsection{}
\begin{align}
n_{Hg(NO_3)_2}=\frac{1000}{200,6+14*2+6*16}=3,1 Mol\\
n_{HgCl2} \rightarrow \frac{1000}{200,6+70}=3,7Mol\\
\end{align}
TODO : Siedepunkterhöhung, dissozieren die Stoffe?

\section{Chemische Kinetik}
\subsection{}
\begin{align}
  v=k [BF_3]^a[NH_3]^b\\
  3/4:0,57=0,57^a \rightarrow a=1\\
  2/5:1,79=1,43*1,25^b \rightarrow b=1\\
  \text{Gesamtordnung:2}\\
  5:k=3,4\\
  \text{Note:Geschwindigkeit von 1 ist wahrscheinlich falsch}
\end{align}

\subsection{}
\begin{align}
  k=A*e^{\frac{-E}{R*T}}\\
  x=e^{\frac{-E}{R}(T_1^-1-T_2^-1)}
  x=88
\end{align}

\subsection{}
\begin{align}
  v_1=k_1 [H_2O_2][I]\\
  v_2=k_2 [IO][H_2O_2]\\
  2 H_2O_2 \rightarrow 2 H_2O + O_2
  \text{Für den Prozess gilt} v_1
\end{align}

\subsection{}
\begin{align}
  k_1[Br_2]=k_2[Br]^2\\
  [Br]=\sqrt{\frac{k_1}{k_2}[Br_2]}
\end{align}

\subsection{}
Da man Metalle sehr dunn aufdampfen kann verbrauchst man weniger -> billiger. Eine größere Oberfläche hat den Vorteil,dass mehr Atome an ihr zu anderen Stoffen reagieren können und dieser damit effektiver ist.

\section{Chemisches Gleichgewicht}
\subsection{}
\begin{align}
  [HF]=\frac{[H][F]}{K_1}\\
  [H_2C_2O_4]=\frac{[H]^2[C_2O_4]}{K_2}\\
  K_{3}=\frac{[F]^2[H_2C_2O_4]}{[HF]^2[C_2O_4]}\\
  K_3=\frac{[F]^2 [H]^2 [C_2O_4] {K_1}^2}{K_2 [H]^2 [F]^2 [C_2O_4]}=\frac{K_1^2}{K_2}\\
  (HO-,0=)C-C(=O,-OH)\\
\end{align}

\subsection{}
Die Konzentration kann durch multiplikation von RT zu Drücken übergeführt werden. Gasgleichung
\begin{align}
  K_1*p_{H_2}*p_{I_2}=p_{HI}^2 \\
  K_2*p_{N_2}*p_{H_2}^3=p_{NH_3}^2\\
  K_3=\frac{p_{HI}^6*p_{N_2}}{p_{NH_3}^2*p_{I_2}^3}=
  \frac{K_1^3*p_{H_2}^3*p_{I_2}^3*p_{N_2}}{K_2*p_{N_2}*p_{H_2}^3*p_{I_2}^3}=  \frac{K_1^3}{K_2}
\end{align}

\subsection{}
\begin{align}
  K_p=\frac{p_{SO_2}^2*p_{O_2}}{p_{SO_3}^2}\\
  p_{{SO_3}_0}=0,5 atm\\
  p_{SO_3}=0,2 atm\\
  p_{{SO_2}_0}=p_{{O_2}_0}=0\\
  P_{SO_2}=0,3 atm\\
  p_{O_2}=0,15 atm\\
  K_p = 0,3375
\end{align}

\subsection{}
\begin{align}
  \frac{p_{SO_2}*p_{O_2}}{p_{S0_3}}=16,4 > K_p
\end{align}
Daraus Folgt, dass der Druck und somit die Teilchenanzahl von $SO_2$ und $O_2$ abnehmen mussen. Die Teilchenanzahl von $S=_3$ muss zunehmen.
\begin{align}
  2SO_2 + O_2 \rightarrow 2SO_3
\end{align}

\subsection{}
\begin{align}
  K_p = \frac{p_{CO}^2}{p_{CO_2}}\\
  p_{CO_2}=1*0,0623=0,0623 atm\\
  p_{CO}=0,9377 atm\\
  K_{p_{850}}=14,1\\
  K_{p_{950}}=79,16\\
  K_{p_{1050}}=268\\ 
\end{align}
Endotherm da mit Steigernder Temperatur K steigt(links nach rechts)

\subsection{}
\begin{align}
  K_p = \frac{p_{PCl_3}*p_{Cl_2}}{p_{PCl_5}}=\frac{x^2}{1.66-x}=0.497\\
  x=\frac{-K_p}{2} \pm \sqrt{1.66*K_p+\frac{K_p^2}{4}} \Rightarrow x=0,69 atm\\
  p_{PCl_3} = p_{Cl_2}=0,69 atm\\
  p_{PCl_5} = 0,97 atm
\end{align}

\section{Säure-Base Gleichgewichte}

\section{Thermodynamik II \& Elektrochemie}
\subsection{}
a.) - da gas $\rightarrow$ flüssig
b.) - da 2 Teilchen -> 1 Teilchen
c.) - da Fest+Gas -> Fest
d.) nicht zu sagen da gleich viele Teichen und alles Gasförmig

\subsection{}
a.) Entropie = 0, Der Ausgangszustand kann wieder hergestellt werden.
b.) Sie ist wie beim Beginn der Prozesses
c.) Beim richtiger Druck/Temperatukombination. Verdampfen/Kondensieren  

\subsection{}
\begin{align}
  Mn^{+7}O_4^{-2} +3 e^- +4 H^+ \rightarrow Mn^{+4}O_2^{-2} +2 H_2^+O^{-2}\\
  Fe^{+2} - e^- \rightarrow Fe^{+3}\\
  Mn^{+7}O_4^{-2} +3 Fe^{+2} +4 H^+ \rightarrow Mn^{+4}O_2^{-2} + 3 Fe^{+3} +2 H_2^+O^{-2}\\
\end{align}

\subsection{}
\begin{align}
  MnO4 + Mn + H \rightarrow MnO_2 + H_2O\\
  Mn^{+7}O_4^{-2} + 3e^- +4H^+ \rightarrow Mn^{+4}O_2^{-2} + 2 H_2O\\
  Mn^{+2} -2e^- + 2H_20 \rightarrow MnO_2 + 4 H_4^+
  2Mn^{+7}O_4^{-2} +8H^+ +3 Mn^{+2} + 6H_20 \rightarrow 2Mn^{+4}O_2^{-2} + 4H_2O + 3MnO_2+12H_4^+\\
2Mn^{+7}O_4^{-2} +3 Mn^{+2} + 2H_20 \rightarrow 5Mn^{+4}O_2^{-2} +4 H^+
\end{align}

\subsection{}
\begin{align}
  Cr_2^{+6}O_7^{-2} + 6e^- + 14H^+ \rightarrow 2 Cr^{+3}+ 7H_2O\\
  C^{-2}H_3^+O^{-2}H^+ -6e^- + H_2O   \rightarrow C^{+4}O_2^{-2} + 6H^+\\
Cr_2^{+6}O_7^{-2} +  14H^+ + C^{+2}H_3^+O^{+2}H^+ + H_2O \rightarrow
 2 Cr^{+3}+ 7H_2O + C^{+4}O_2^{-2} + 6H^+\\
Cr_2^{+6}O_7^{-2}  + C^{+2}H_3^+O^{+2}H^+ 8 H^+ \rightarrow
 2 Cr^{+3}+ 6H_2O + C^{+4}O_2^{-2} 
\end{align}


\end{document}


